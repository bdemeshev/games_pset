% gt_introduction

Задачи взяты из различных источников, по возможности, указан автор.

В коллекцию (Леша разрешил!) перекочевали задачи из «Большого Задачника Игр» (С. Коковин, А. Тонис, А. Савватеев и др.). Эти задачи отмечены так «БЗИ».

Если есть вопросы по этой коллекции --- boris.demeshev@gmail.com




%ver 04.06.07 Все сброшено в tex с потерей форматирования и рисунков \par
%ver 06.06.07 Восстановлено кое-как форматирование \par
%ver 18.06.07 Восстановление картинок к играм и оформления продолжается \par
%ver 18.11.07 Остались деревья: rayles, к берегу водохранилища, несколько деревьев в лесу
%ver 15.05.08 Ковбои \par
%ver 05.06.08 Rayles таки сдан! Дележ рутинной работы, ссылки для дележа пирога \par

\subsection{Цитаты}

Воцарилось молчание, во время которого Карлсон дожёвывал свой шоколад. Потом он сказал:

- Но раз ты такой лакомка, такой обжора, лучше всего будет по-братски поделить остатки. У тебя ещё есть конфеты?

Малыш пошарил в карманах.

- Вот, три штуки. - И он вытащил два засахаренных орешка и один леденец.

- Три пополам не делится, - сказал Карлсон, - это знают даже малые дети. - И, быстро схватив с ладони Малыша леденец, проглотил его. - Вот теперь можно делить, - продолжал Карлсон и с жадностью поглядел на оставшиеся два орешка: один из них был чуточку больше другого. - Так как я очень милый и очень скромный, то разрешаю тебе взять первому. Но помни: кто берёт первым, всегда должен брать то, что поменьше, - закончил Карлсон и строго взглянул на Малыша.

Малыш на секунду задумался, но тут же нашёлся:

- Уступаю тебе право взять первым.

- Хорошо, раз ты такой упрямый! - вскрикнул Карлсон и, схватив больший орешек, мигом засунул его себе в рот.

Малыш посмотрел на маленький орешек, одиноко лежавший на его ладони.

- Послушай, - сказал он, - ведь ты же сам говорил, что тот, кто берёт первым, должен взять то, что поменьше.

- Эй ты, маленький лакомка, если бы ты выбирал первым, какой бы орешек ты взял себе? - Можешь не сомневаться, я взял бы меньший, - твёрдо ответил Малыш.

- Так что ж ты волнуешься? Ведь он тебе и достался!

Малыш вновь подумал о том, что, видимо, это и есть то самое разрешение спора словами, а не кулаками, о котором говорила мама.
{\it Карлсон, который живёт на крыше,  А. Линдгрен}
\vspace{0.5cm}


There are two important concepts in economics. The first is «Buy low, sell high», which is self-explanatory. The second is opportunity cost, the highest valued alternative that must be sacrificed to attain something or otherwise satisfy a want. I discovered this concept as an undergraduate at Caltech. I was never very in to computer games, but I found myself randomly playing tetris. Out of the blue I was struck by a revelation: «I could be having sex right now.» I haven't played a computer game since.\par
{\it Introduction to Methods of Applied Mathematics, Sean Mauch}\par
\vspace{0.5cm}

Бюджетное ограничение следует называть принципом кота Матроскина: «Чтобы продать что-нибудь ненужное, надо сначала купить что-нибудь ненужное».\par
{\it идея Юры Автономова}\par
\vspace{0.5cm}
Правильно хватать самый маленький кусок торта! Его можно съесть раньше, чем сестры доедят свои куски, и тогда успеешь взять еще и второй!\par
{\it по мотивам «Делим по справедливости», Брамс}\par

\vspace{0.5cm}

In mastering the material in this book, you are going to have to do a lot of work. This will consist mainly of chewing a pencil or pen as you struggle to do some sums. Maths is like that. Hours of your life will pass doing this, when you could be watching the X-files or playing basketball, or whatever.\par
{\it Michael D. Alder, An Introduction to  Complex Analysis for Engineers}\par
\vspace{0.5cm}

Well of course I didn't do any at first ... then someone suggested I try just a little sum or two, and I thought «Why not? ... I can handle it». Then one day someone said «Hey, man, that's kidstuff - try some calculus» ... so I tried some differentials ... then I went on to integrals ... even the occasional volume of revolution ... but I can stop any time I want to ... I know I can. OK, so I do the odd bit of complex analysis, but only a few times ... that stuff can really screw your head up for days ... but I can handle it ... it's OK really ... I can stop any time I want ...\par
{\it tim@bierman.demon.co.uk (Tim Bierman)}\par

\vspace{0.5cm}

\subsection{О чём это всё}

Задачник не вызывает сонливости и не имеет противопоказаний.\par
В случае крайне маловероятной посадки на воду задачник может быть использован в качестве спасательного средства. \par
Задачник оборудован тремя аварийными выходами: в передней, средней и хвостовой частях.\par
Отпускается без рецепта.\par
Срок годности не ограничен.\par
Не содержит мелких деталей, которые могут быть проглочены детьми до 3-х лет.\par
\vspace{0.5cm}
{\bf }Предисловие:\par
Задачи упорядочены только по типу!\par

Условия задач кроме составителя комментировал Тигр - тотем теории игр.\par


Краткий словарь:\par
Nash Equilibrium, NE - равновесие по Нэшу. В задачнике этот термин используется максимально широко, в том числе и для игр с неполной информацией.\par
Subgame Perfect Nash Equilibrium, SPNE - равновесие по Нэшу,  совершенное на подыграх\par
Weak Sequential Equilibrium - слабое секвенциальное равновесие \par
Sequential Equilibrium, SE - секвенциальное равновесие\par
Common knowledge of information - всеобщность знания\par
Correlated Equilibrium - Коррелированное равновесие\par

Многие задачи можно решать, руководствуясь только здравым смыслом. Т.е. если Вы все-таки так и не поняли, что такое равновесие по Нэшу, попытайтесь ответить на вопрос «Как бы я играл в эту игру?»\par
По умолчанию предполагается, что все сказанное в условии является всеобщим знанием.\par

Что означают пометки у задач?\par
$[$Т$]$ - Трудная задача\par
$[$О$]$ - Особенная задача. У особенной задачи может быть смешное условие, а ее решение может кардинально изменить геополитическую обстановку в мире.




%Маршруты для туристов (нужно разработать):

%$[$Сюда также что-то типа: эти задачи (список) предлагались в курсе ВШЭ, т.е. несколько маршрутов для самостоятельного изучения$]$

%«Последний двоечник». Задачи для тех, кто хочет гарантировать себе один или, если повезет, два балла из десяти.

%«PG-13» Parents strongly cautioned. Some material may be inappropriate for children under 13.\par

%«NC-17» No one 17 and under admitted. May contain explicit sex scenes, and/or scenes of excessive violence.\par

%«Для тех, кто и вправду крут». {\it Тигр:  По-моему, это задачи, которые составитель задачника не смог решить сам...} \par

%«Для прессы» {\it Тигр: Подполковник Киселевич говорил, что любая аудитория делится на три части. Спереди сидят лошади, они все время  пашут, посередине сидят бездельники, они ничего не делают, а сзади сидит пресса, она ждет сенсаций.}\par
